\documentclass{ees}

\begin{document}

\eesTitlePage

\eesCriticalReport{
  1 & –   & –    & The da capo of the \textit{Requiem} is written out
                   in all parts except vl 1/2. \\
  \midrule
  2 & 18  & vl 1 & 5th \eighthNote\ in \B1: g′8 \\
  \midrule
  3 & 143 & fag  & bar in \B1: \wholeNoteRest \\
    & 148 & vl 2 & 2nd \eighthNote\ in \B1: c″8 \\
    & 172 & cnto, vl 1, S & 3rd \quarterNote\ in \B1: \flat d″4
                   (but cf. bass figures!) \\
    & 218 & vl 2  & 2nd \halfNote\ in \B1: g′4–a′4 \\
    & 223 & fag   & bar in \B1: \wholeNoteRest \\
    & 245 & trb 1, A & 2nd \halfNote\ in \B1: \sharp f′4 \\
    & 246 & fag   & bar in \B1: \wholeNoteRest \\
    & 286 & fag, org & bar in \B1: d1. \\
    & 291–301 & fag & in \B1 unison with org \\
  \midrule
  4 & –   & –     & The da capo of the \textit{Quam olim} is written out
                    in A solo, vl 1/2, org, and mdc. \\
    & 7   & trb 1, A & 1st \halfNote\ in \B1: f′8.–f′16–f′4 \\
    & 40  & org   & last \quarterNote\ in \B1: \flat e′4 \\
    & 56  & trb 2, T & The last \quarterNote\ in \B1 has been corrected
                    to f′4, but the (presumably original) \flat e′4
                    would be a more appropriate choice. \\
  \midrule
  5 & 14  & trb 2 & 6th/7th \quarterNote\ in \B1: \flat e′4 \\
    & 24  & vl 2  & 2nd \eighthNote\ in \B1: f′8 \\
  \midrule
  7 & –   & –    & The da capo of the \textit{Requiem} is written out
                   in S, A, T, B solo, clno 1/2, vl 1/2, and org. \\
    & 22  & vl 2 & 2nd \halfNote\ in \B1: d′4–\sharp e′4 \\
    & 55  & vl 1 & 2nd \halfNote\ in \B1: d″4–d″4 \\
}

\eesToc{}

\eesScore

\end{document}
